\documentclass[11pt]{article}
\usepackage{fullpage}
\usepackage{fancyhdr}
\usepackage{epsfig}
\usepackage{algorithm}
\usepackage[noend]{algorithmic}
\usepackage{amsmath,amssymb,amsthm}
\usepackage{graphicx}



% FILL IN THE SPECIFICS OF EACH HOMEWORK HERE
\newcommand{\course}{Music 421a}
\newcommand{\semester}{Winter 2012}
\newcommand{\name}{Audio Applications of the FFT}
\newcommand{\hwk}{Homework \#6 Solutions}
\newcommand{\student}{Hunter McCurry}


%%
% The following are definitions so that you can use numbered lemmas, claims, etc.
%%
\newtheorem{lemma}{Lemma}
\newtheorem*{lem}{Lemma}
\newtheorem{definition}{Definition}
\newtheorem{notation}{Notation}
\newtheorem*{claim}{Claim}
\newtheorem*{fclaim}{False Claim}
\newtheorem{observation}{Observation}
\newtheorem{conjecture}[lemma]{Conjecture}
\newtheorem{theorem}[lemma]{Theorem}
\newtheorem{corollary}[lemma]{Corollary}
\newtheorem{proposition}[lemma]{Proposition}

%%
% The following are definitions so that you can use some shorthand with deltas and such
%%
\newcommand{\deltahat}{\hat{\delta}}
\newcommand{\Deltahat}{\hat{\Delta}}
\newcommand{\righti}{\stackrel{i}{\rightarrow}}
\newcommand{\rights}{\stackrel{*}{\rightarrow}}
\newcommand{\righto}{\stackrel{1}{\rightarrow}}
\newcommand{\rightn}{\stackrel{n}{\rightarrow}}
\newcommand{\rightnp}{\stackrel{n+1}{\rightarrow}}
\newcommand{\rightip}{\stackrel{i+1}{\rightarrow}}
\newcommand{\rightk}{\stackrel{k}{\rightarrow}}
\newcommand{\rightln}{\stackrel{\leq n}{\rightarrow}}



%%% You can ignore the following stuff, it's just for formatting purposes
\textheight=8.6in
\setlength{\textwidth}{6.44in}
\addtolength{\headheight}{\baselineskip} 
% enumerate uses a), b), c), ...
\renewcommand{\labelenumi}{\alph{enumi})}
% Sets the style for fancy pages (namely, all but first page)
\pagestyle{fancy}
\fancyhf{}
\renewcommand{\headrulewidth}{0.0pt}
\renewcommand{\footrulewidth}{0.4pt}
% Changes style of plain pages (namely, the first page)
\fancypagestyle{plain}{
  \fancyhf{}
  \renewcommand\headrulewidth{0pt}
  \renewcommand\footrulewidth{0.4pt}
  \renewcommand{\headrule}{}
  }
% Changes the title box on the first page
\renewcommand\maketitle{
\begin{center}
\begin{tabular*}{6.44in}{l @{\extracolsep{\fill}}c r}
\bfseries  &  & \bfseries \course ~\semester \\
\bfseries&  & \bfseries  \hwk  \\
\bfseries   &   &  \bfseries \student \\ 
\end{tabular*}
\end{center} }


%%
%%
%% THE REAL STUFF STARTS HERE
%%
%%
\begin{document}
\maketitle
\thispagestyle{plain}

\subsection*{Question 1}
Because we know $S(f) \propto 1/f$ (that is, $S(f) = K/f$), and because the average power within a frequency band is obtained by integrating the sample power spectral density across that band,

\begin{align*}
P &= \! \int_{\omega}^{2\omega} S(f)\,df \\
&= K \! \int_{\omega}^{2\omega} 1/f\,df \\
&= K [ \log{2\omega} - \log{\omega}] \\
&= K \log{2}
\end{align*}

\subsection*{Question 2}
\begin{enumerate}
\item Here $s(t) = 2\delta(t)$, so:
\begin{align*}
r(t) &= (s * h)(t) + n(t) \\
&= (2\delta * h)(t) + n(t) \\
&= 2(\delta * h)(t) + n(t). \\
\end{align*}
Normalizing (multiplying by $1/2$), we get:
\begin{align*}
&= (\delta * h)(t) + \frac{n(t)}{2}. \\
\end{align*}
We know the mean of $n(t)$ is zero and the variance of $n(t)$ is $\sigma^2$ so the variance of $\frac{n(t)}{2}$ will be given by:
\begin{equation}
E\left\{ \left| \frac{n(t)}{2} - 0 \right|^2\right\} = E\left\{ \left| \frac{n(t)}{2}\right|^2\right\} = \frac{\sigma^2}{4}\\
\end{equation}

\item Now we have $s(t) = 2\delta(t)$ and two responses, $r_1(t)$ and $r_2(t)$:
\begin{align*}
r_1(t) &= (s * h)(t) + n_1(t), \\
r_2(t) &= (s * h)(t) + n_2(t), \\
r_1(t) + r_2(t) &= 2(s * h)(t) + \frac{n_1(t)}{2} + \frac{n_2(t)}{2}
\end{align*}

Because we know that the variance of $\frac{n(t)}{2}$ is $\frac{\sigma^2}{4}$ (from above), and because $n_1(t)$ and $n_2(t)$ will not add coherently,  our new variance is $2\frac{\sigma^2}{4} = \frac{\sigma^2}{2}$.

\item We have $a(t)$ and $b(t)$ such that $(a \star b)(t) + (b \star b)(t) = 2L\delta(t)$. We have
\begin{align*}
r_a(t) &= (a * h)(t) + n_a(t), \\
r_b(t) &= (b * h)(t) + n_b(t), \\
(a \star r_a)(t) &= a \star ((a * h)(t) + n_a(t)), \\
(b \star r_b)(t) &= b \star ((b * h)(t) + n_b(t)), \\
(a \star r_a)(t) + (b \star r_b)(t) &= a \star (a * h)(t) + a \star n_a(t) + b \star (b * h)(t) + b \star n_b(t)\\
&= ((a \star b)(t) + (b \star b)(t)) * h(t) + (n_a \star a)(t) + (n_b \star b)(t) \\
&= 2L\delta * h(t) + (n_a \star a)(t) + (n_b \star b)(t)
\end{align*}

\end{enumerate}

\subsection*{Question 3}
\begin{enumerate}
\item 
\item
\item
\end{enumerate}


\end{document}
