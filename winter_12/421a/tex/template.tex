\documentclass[11pt]{article}
\usepackage{fullpage}
\usepackage{fancyhdr}
\usepackage{epsfig}
\usepackage{algorithm}
\usepackage[noend]{algorithmic}
\usepackage{amsmath,amssymb,amsthm}



% FILL IN THE SPECIFICS OF EACH HOMEWORK HERE
\newcommand{\course}{CSCI 383}
\newcommand{\semester}{Fall 2009}
\newcommand{\name}{Theory of Computation}
%%%
%%%
%%% PLEASE FILL OUT YOUR NAME AND THE HWK NUMBER
%%%
%%%
\newcommand{\hwk}{Homework \#? Solutions}
\newcommand{\student}{Put Your Name Here}


%%
% The following are definitions so that you can use numbered lemmas, claims, etc.
%%
\newtheorem{lemma}{Lemma}
\newtheorem*{lem}{Lemma}
\newtheorem{definition}{Definition}
\newtheorem{notation}{Notation}
\newtheorem*{claim}{Claim}
\newtheorem*{fclaim}{False Claim}
\newtheorem{observation}{Observation}
\newtheorem{conjecture}[lemma]{Conjecture}
\newtheorem{theorem}[lemma]{Theorem}
\newtheorem{corollary}[lemma]{Corollary}
\newtheorem{proposition}[lemma]{Proposition}
\newtheorem*{rt}{Running Time}

%%
% The following are definitions so that you can use some shorthand with deltas and such
%%
\newcommand{\deltahat}{\hat{\delta}}
\newcommand{\Deltahat}{\hat{\Delta}}
\newcommand{\righti}{\stackrel{i}{\rightarrow}}
\newcommand{\rights}{\stackrel{*}{\rightarrow}}
\newcommand{\righto}{\stackrel{1}{\rightarrow}}
\newcommand{\rightn}{\stackrel{n}{\rightarrow}}
\newcommand{\rightnp}{\stackrel{n+1}{\rightarrow}}
\newcommand{\rightip}{\stackrel{i+1}{\rightarrow}}
\newcommand{\rightk}{\stackrel{k}{\rightarrow}}
\newcommand{\rightln}{\stackrel{\leq n}{\rightarrow}}



%%% You can ignore the following stuff, it's just for formatting purposes
\textheight=8.6in
\setlength{\textwidth}{6.44in}
\addtolength{\headheight}{\baselineskip} 
% enumerate uses a), b), c), ...
\renewcommand{\labelenumi}{\alph{enumi})}
% Sets the style for fancy pages (namely, all but first page)
\pagestyle{fancy}
\fancyhf{}
\renewcommand{\headrulewidth}{0.0pt}
\renewcommand{\footrulewidth}{0.4pt}
% Changes style of plain pages (namely, the first page)
\fancypagestyle{plain}{
  \fancyhf{}
  \renewcommand\headrulewidth{0pt}
  \renewcommand\footrulewidth{0.4pt}
  \renewcommand{\headrule}{}
  }
% Changes the title box on the first page
\renewcommand\maketitle{
\begin{center}
\begin{tabular*}{6.44in}{l @{\extracolsep{\fill}}c r}
\bfseries  &  & \bfseries \course ~\semester \\
\bfseries&  & \bfseries  \hwk  \\
\bfseries   &   &  \bfseries \student \\ 
\end{tabular*}
\end{center} }




%%
%%
%% THE REAL STUFF STARTS HERE
%%
%%
\begin{document}
\maketitle
\thispagestyle{plain}


%%% PLEASE PLACE THE HONOR CODE AND YOUR NAME/SIGNATURE HERE
\noindent 


\subsection*{Question 1}
% Your solution to Q1 goes here.
% Your solution to Q1 goes here.
Let $P(h)$ be the statement that
\begin{claim}
$P(h)$ is true for all $h \geq 0$.
\end{claim}

\begin{proof}
We prove the claim by induction on $h$.
\begin{description}
\item[BC:] As a basis, consider $h=?$.
\item[IS:] Suppose $P(h)$ is true for all ...
\item[IH:] Consider $P(?)$.
\end{description}
Therefore $P(h)$ is true for all $h \geq 0$.
\end{proof}



\subsection*{Question 2}
% Your solution to Q2 goes here.

\begin{enumerate}
\item $A \cup B = \{ \}$.
\item $A \cap B = \{ \}$.
\item $A \times B = \{ \}$.
\item $2^B = \{ \}$.
\end{enumerate}


\subsection*{Question 3}
% Your solution to Q3 goes here.

\begin{enumerate}
\item Define the relation $\approx$ on the $\mathbb{Z}$ as follows: for $x, y \in \mathbb{Z}$, 
\[ x \approx y ~\stackrel{def}{\iff}~ \deltahat(x,a)=\deltahat(y,b)\]
\item Define the relation $\approx$
\end{enumerate}




\subsection*{Question 4}
% Your solution to Q4 goes here.

\begin{claim}
$R \cup (S \cap T) = (R \cup S) \cap (R \cup T)$ for all sets $R,S$ and $T$.
\end{claim}

\begin{proof}
There are two cases:
\begin{itemize}
\item[i)]
\item[ii)]
\end{itemize}
\end{proof}



%% I can force a new page as follows
\newpage
%and I may want to add a little vertical space
\vspace*{2pt}




%% Whatever is between the two &'s gets lines up. In this case, the equal signs.
\noindent We can define the transition function as
\begin{eqnarray*}
\delta(0, a) & = & 1,\\
\delta( 1, a) & = & 2,\\
\delta(2, \varepsilon) & = & \delta(3,a) ~=~3~.
\end{eqnarray*}


\noindent A \emph{deterministic finite automaton} (DFA) is a structure 
\[ M = (Q, \Sigma, \delta, s, F), \]
where
\begin{itemize}
\item $Q$ is a finite set of \emph{states};
\item $\Sigma$ is a finite alphabet;
\item $\delta: Q \times \Sigma \rightarrow Q$ is the transition function;
\item $s \in Q$ is the start state;
\item $F \subseteq Q$  of \emph{final states}.
\end{itemize}






%%% PLEASE ERASE THE STUFF YOU DON'T USE FROM YOUR SUBMISSION


%% To put in blank space (vertical space), use the following command.
%% You can adjust the number of points to suit your tastes.
%% negative points will shift stuff up, if you want
\vspace*{50pt}

%%% Possibly useful stuff
\noindent Possibly useful stuff.


% Regular text is roman font.
\noindent Blahblahblah in roman font, indented by default.\\ 

% and \\ ends a line. 
 %Note that a new line is automatically indented 
 %(unless it is the first line in your subsection / question)
 
\noindent \textbf{Blahalala} in bold font, unindented.\\
 
\noindent \emph{blahblahblah} in italics. \\

\noindent \verb+blah blahblah blah+ \texttt{blahblahba} in true-type font.\\ 

\begin{lemma}
something happens $\iff$ something else happens.
\end{lemma}

\begin{proof}And I'm proving it here. I'll start with the forward direction, and then the backwards.\medskip

\noindent $(\Rightarrow)$ Forward direction proven here.\medskip

\noindent $(\Leftarrow)$ Backwards direction proven here.\end{proof}


Now I will put up a system of equations all lined up n' purdy.
\begin{eqnarray*}
LHS & = & n(n-1) \\
& = & 2n(n-1)/2 \\
& \stackrel{(*)}{=} & RHS
\end{eqnarray*}
%% Whatever is between the two &'s gets lines up. In this case, the equal signs.

Oh, and this is nifty, although maybe a bit confusing...
\begin{eqnarray*}
e_{ij} & = & \left\{ \begin{tabular}{ll} $\{v_i, v_j\}$ & if edges are undirected, \\
						       $(v_i, v_j)$ & otherwise. \end{tabular} \right.
\end{eqnarray*}




\end{document}
